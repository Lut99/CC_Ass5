\documentclass[12pt]{article}

\usepackage{minted}
\usepackage{tcolorbox}
\usepackage{xcolor}
\usepackage{hyperref}

\newcommand*{\link}[2]{\href{#1}{\color{blue}\textbf{\textit{#2}}}}

\title{C++ Programmeermethoden Assignment5}
\author{Bas Terwijn}
\date{\today}

\begin{document}
\maketitle

\section{Introduction}
Assignment5 replaces the 2020 C++ Programmeermethoden exams because of
ongoing online teaching due to Corona virus measures and will
determine your grade for this course. In this assignment you are asked
to modify and extend the
\link{https://bitbucket.org/bterwijn/virusgame} { VirusGame }
software.
     
\section{Tasks}
You are asked to do the following tasks but first install and run
VirusGame on your computer using the installation instructions
provided and read the documentation and code to get somewhat
familiar with it.

\subsection{task1: Polymorphism}
Currently in VirusGame.cpp all units are stored in a static array:

\begin{tcolorbox}
\begin{minted}{c++}
Virus units[max_nr_units];
\end{minted}
\end{tcolorbox}

But we want to be able to add other classes as units besides only the
``Virus'' class. In addition we want to handle the ``player'' object as
just another unit so the code gets simpler. Therefore change the
``units'' array so that units of different types can be added to it
using dynamic polymorphism.

\subsection{task2: Avoid Memory Leaks}
With polymorphism you often will dynamically allocate memory when you
instantiate objects using the ``new'' keyword. Avoid memory leaks by
de-allocating the memory when it is no longer needed.

\subsection{task3: RAII}
With \link{https://en.cppreference.com/w/cpp/language/raii}{Resource
  Acquisition Is Initialization (RAII)} you can make sure you, and
possible other people that later might use your code, won't forget to
release or de-allocate any resources such as memory. This is done by
putting the code that releases the resource in a destructor that is
automatically called when an object goes out of scope. Use RAII to
release any resources in your code, for example the memory allocated
for units.

\subsection{task4: STL Containers}
The modern \link{http://www.cplusplus.com/reference/stl/}{Standard
  Template Library containers} are the preferred data structures to
use. Prefer std::vector over a static array [] as it can grow to
arbitrary size, it knows its own size, doesn't decay to a pointer when
passed to a function, and has only little additional overhead compared
to a static array. Therefore replace any static array in your code
(for example: Virus units[max\_nr\_units];) with a std::vector and
prefer stl containers if you choose to add other data structures.

\subsection{task5: STL Algorithms}
\link{https://isocpp.github.io/CppCoreGuidelines/CppCoreGuidelines\#Res-lib}
{ES.1 of C++ Core Guidelines} recommends using the standard library
over ``handcrafted code''.  Therefore use as much as possible the
functions defined in the
\link{https://en.cppreference.com/w/cpp/algorithm} {STL Algorithms
  Library} instead of for example raw for-loops. For a gentle
introduction to STL Algorithms see the
\link{https://www.youtube.com/watch?v=2olsGf6JIkU} {``105 STL
  Algorithms in Less Than an Hour''} talk by Jonathan Boccara.

\subsection{task6: Avoid Duplicate Code}
Avoid having duplicate code or expressions, or said differently, don't
repeat yourself (DRY). The Virus::step() function is currently already
a duplicate of Player::step(). Find a good way to avoid that and other
duplication.

\subsection{task7: Your Own Creative Extension}
The VirusGame is not yet finished. Extend it so it has interesting
game play. Maybe the player has to avoid touching the viruses, or
shoot them, or bump into them to bounce them into an anti-virus
unit. Maybe also add some special effects like explosions or tire/skid
marks or keep a score. The more creative the better, make it fun. You
are free to change anything in the provided source code. Write a short
description of your extension with references to source code at the
bottom of the README.md file just so that I don't miss anything when
grading your submission.

\section{Grading}
Your grade will follow from which tasks you complete to a satisfactory
level:
\begin{center}
\begin{tabular}{ |l|c| } 
  \hline
  \textbf{task}               &  \textbf{points}\\ \hline
  task1: Polymorphism         &               2 \\
  task2: Avoid Memory Leaks   &               1 \\
  task3: RAII                 &               1 \\
  task4: STL Containers       &               1 \\
  task5: STL Algorithms       &               1 \\
  task6: Avoid Duplicate Code &               1 \\
  task7: Creative Extension   &               3 \\
  \hline
\end{tabular}
\end{center}

Points will be deducted if your code is not ``simple'' such as
described by Kate Gregory in her
\link{https://www.youtube.com/watch?v=O50qTuM5OT0} {``Simplicity: not
  just for beginners''} talk, watch it!

\section{Rules}
You are not allowed to share code with other students, if we detect
(manually or with plagiarism checkers) that different submissions have
similar structure I will have to report that to the examination
board. See the UvA ``Fraude en plagiaat regeling'' for more details.

Therefore if you optionally choose to fork the VirusGame git
repository so you can use git to track and backup your changes, then
make sure the repository is private otherwise you could get accused of
plagiarism if someone copies your code. Bitbucket allows you to make
repositories private if you use your ``@uva.nl'' email address for
your profile.

\section{Submission}
Submit your code as a zip/tar of the whole VirusGame project before
the deadline on May 31 23:59 on Canvas. Remove the compiled
executables and other derivatives that I don't need to compile your
code in order to reduce the size. If you use other dependencies
(additional libraries) describe them so I can easily install those
before compiling your code. Double check that your submission contains
all required files.

\end{document}